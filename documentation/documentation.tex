\documentclass{article}

% set font encoding for PDFLaTeX, XeLaTeX, or LuaTeX
\usepackage{ifxetex,ifluatex,amsmath,amsfonts,amssymb,graphicx}
\if\ifxetex T\else\ifluatex T\else F\fi\fi T%
  \usepackage{fontspec}
\else
  \usepackage[T1]{fontenc}
  \usepackage[utf8]{inputenc}
  \usepackage{lmodern}
\fi

\usepackage{hyperref}

% Nashir's includes
\usepackage{commath}
% end includes

\title{Final Project Documentation}
\author{Nashir Janmohamed}
\date{Due 11:59 pm, Monday, May 25}

% Enable SageTeX to run SageMath code right inside this LaTeX file.
% http://doc.sagemath.org/html/en/tutorial/sagetex.html
% \usepackage{sagetex}

% Enable PythonTeX to run Python – https://ctan.org/pkg/pythontex
% \usepackage{pythontex}

\begin{document}
\maketitle


\section{Recurrence Relations/Dynamic Programming}
\subsection{Bell numbers}
The Bell numbers represent the number of ways to count partitions of (or equivalently equivalence relations on) an $n$ element set. The $n$-th bell number is given by the recurrence
\[B_n = \sum_{k=1}^{n} {n-1 \choose k-1} B_{n-k}\]
for $n \ge 0$.
\subsection{Catalan numbers}
The Catalan numbers form a sequence of natural numbers that occur in various counting problems, often involving recursively-defined objects. They can be expressed by the recurrence relation
\[C_{n+1} = \sum_{i=0}^{n}C_iC_{n-1}\]
for $n \ge 0$.

\noindent Their closed form is given by
\[{2n \choose n} - {2n \choose n+1 }\]
\subsection{Fibonacci numbers}
The Fibonacci numbers, commonly denoted $F_n$, form a sequence such that each number is the sum of the two preceding ones, with $F_0 = 0$, $F_1 = 1$, and the recurrence given by
\[F_n = F_{n-1}+F_{n-2}\]
for $n > 1$.
\subsection{Stirling numbers of the first kind}
\subsection{Stirling numbers of the second kind}

\section{Permutations and Combinations}
\subsection{Combinations without repetition}
A combination without repetition is a selection of items from a collection, such that the order of selection does not matter. A $k$-combination of an $n$ element set $S$ is a subset of $k$ distinct elements. The number of $k$-combinations is equal to the binomial coefficient given by
\[{n \choose k} = \frac{n!}{(n-k)!\,k!}\]

\subsection{Permutations without repetition}
$k$-permutations of $n$ are the different ordered arrangements of a $k$-element subset of an $n$-set. This number is given by
\[P(n,k) = n \cdot (n-1) \cdot (n-2) \cdot \dots (n-k+1) = \frac{n!}{(n-k)!}\]
\subsection{Combinations without repetition}
A $k$-combination with repetitions allowed is a sequence of $k$ not necessarily distinct elements of $S$, where order is not taken into account, i.e. the number of ways to sample $k$ elements from a set of $n$ elements allowing for duplicates but disregarding different orderings. Using the \emph{stars and bars method}, it can be shown that this number is given by
\[{n+k-1 \choose k}\]

\subsection{Permutations with repetition}
Permutations with repetition are ordered arrangements of $k$ elements from a set $S$ with $n$ elements where repetition is allowed. The number of permutations with repetition of size $k$ is simply $k^n$ (except if $k > n$, where the result is 1).
\subsection{Generate permutations of a string}
\subsection{Generate all bit strings of length n}

\section{Relations}
\subsection{\# of relations}
A relation on an $n$ element set $S$ is a subset of $S \times S$, or equivalently, an element of the power set of $S \times S$. There are
\[2^{\,\abs{S}\,\abs{S}}=2^{n^2}\]
such subsets.
\subsection{\# of transitive relations}
There is no known closed formula for counting the number of transitive relations. The (perhaps inefficient) approach taken in this algorithm is as follows
\begin{enumerate}
\item[(1)] Generate all possible relations for an $n$ element set (given by the power set of the cartesian product of the set $\{1,2,3,\dots,n\}$)
\item[(2)] For each relation generated in (1), check that for each $(a,b)$, if there is a point of the form $(b,c)$, then $(a,c)$ must be in the relation
\end{enumerate}
\subsection{\# of (ir)reflexive relations}
A relation is reflexive if all elements are related to themselves, or equivalently, all entries on the main diagonal of the matrix representation of the relation must be 1. There are $n^2$ entries in the matrix and $n$ entries on the main diagonal. For the remaining $n^2-n$ off diagonal entries, the ordered pair may or may not be in the relation. Thus, there are
\[2^{n^2-n}\]
reflexive relations. The argument for irreflexive relations is the same, with the exception that all entries on the main diagonal of the matrix representation of the relation must be 0.
\subsection{\# of symmetric relations}
A relation $R$ is symmetric if for all $(a,b)$ that are in $R$, $(b,a)$ is also in $R$. Each element on the diagonal may or may not be related to itself, and similarly for all the ${n \choose 2}$ two element subsets (with distinct elements). Thus, there are
\[2^{{n \choose 2} + n} = 2^{\frac{n(n+1)}{2}}\]
symmetric relations on a set with $n$ elements.
\subsection{\# of antisymmetric relations}
A relation $R$ is antisymmetric if for all $(a,b)$ that are in $R$, if $(b,a)$ is in $R$, then $a=b$. There are two choices for every element on the diagonal. For the remaining ${n \choose 2}$ two element subsets (with distinct elements) with elements $a$ and $b$, either $(a,b) \in R$ and $(b,a) \not \in R$, $(a,b) \not \in R$ and $(b,a) \in R$, or $(a,b) \not \in R$ and $(b,a) \not \in R$, so there are 3 choices for each two element subset. Thus, there are
\[2^n3^{{n \choose 2}} = 2^n3^{\frac{n(n-1)}{2}}\]
antisymmetric relations on a set with $n$ elements.
\subsection{\# of equivalence relations}
A relation $R$ on a set $A$ is an equivalence relation if it is reflexive, symmetric, and transitive. For each $a \in A$, the equivalence class of $a$ is given by $[a] = \{x \mid xRa\}$. The equivalence classes form a partition of $A$, and so the number of equivalence relations on a set $S$ is given by the number of partitions of a set $S$. So, this number is equivalent to the Bell number (number of partitions/equivalence relations for an $n$ element set) which we can compute directly.
\[B_n = \sum_{k=1}^{n} {n-1 \choose k-1} B_{n-k}\]
for $n \ge 0$.

\section{Sets}
\subsection{Generate power set}
\subsection{Generate cartesian product}

\section{Isomorphisms}
maybe total orders?

\section{Default}
No documentation provided.


\end{document}
