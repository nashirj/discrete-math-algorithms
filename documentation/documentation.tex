\documentclass{article}

% set font encoding for PDFLaTeX, XeLaTeX, or LuaTeX
\usepackage{ifxetex,ifluatex,amsmath,amsfonts,amssymb,graphicx}
\if\ifxetex T\else\ifluatex T\else F\fi\fi T%
  \usepackage{fontspec}
\else
  \usepackage[T1]{fontenc}
  \usepackage[utf8]{inputenc}
  \usepackage{lmodern}
\fi

\usepackage{hyperref}

% Nashir's includes
% end includes

\title{Final Project Documentation}
\author{Nashir Janmohamed}
\date{Due 11:59 pm, Monday, May 25}

% Enable SageTeX to run SageMath code right inside this LaTeX file.
% http://doc.sagemath.org/html/en/tutorial/sagetex.html
% \usepackage{sagetex}

% Enable PythonTeX to run Python – https://ctan.org/pkg/pythontex
% \usepackage{pythontex}

\begin{document}
\maketitle


\section{Recurrence Relations/Dynamic Programming}
\subsection{Bell numbers}
The Bell numbers represent the number of ways to count partitions of (or equivalently equivalence relations on) an $n$ element set. The $n$-th bell number is given by the recurrence
\[B_n = \sum_{k=1}^{n} {n-1 \choose k-1} B_{n-k}\]
for $n \ge 0$.
\subsection{Catalan numbers}
The Catalan numbers form a sequence of natural numbers that occur in various counting problems, often involving recursively-defined objects. They can be expressed by the recurrence relation
\[C_{n+1} = \sum_{i=0}^{n}C_iC_{n-1}\]
for $n \ge 0$.

\noindent Their closed form is given by
\[{2n \choose n} - {2n \choose n+1 }\]
\subsection{Fibonacci numbers}
The Fibonacci numbers, commonly denoted $F_n$, form a sequence such that each number is the sum of the two preceding ones, with $F_0 = 0$, $F_1 = 1$, and the recurrence given by
\[F_n = F_{n-1}+F_{n-2}\]
for $n > 1$.
\subsection{Stirling numbers of the first kind}
\subsection{Stirling numbers of the second kind}

\section{Permutations and Combinations}
\subsection{Combinations without repetition}
A combination without repetition is a selection of items from a collection, such that the order of selection does not matter. A $k$-combination of an $n$ element set $S$ is a subset of $k$ distinct elements. The number of $k$-combinations is equal to the binomial coefficient given by
\[{n \choose k} = \frac{n!}{(n-k)!\,k!}\]

\subsection{Permutations without repetition}
$k$-permutations of $n$ are the different ordered arrangements of a $k$-element subset of an $n$-set. This number is given by
\[P(n,k) = n \cdot (n-1) \cdot (n-2) \cdot \dots (n-k+1) = \frac{n!}{(n-k)!}\]
\subsection{Combinations without repetition}
A $k$-combination with repetitions allowed is a sequence of $k$ not necessarily distinct elements of $S$, where order is not taken into account, i.e. the number of ways to sample $k$ elements from a set of $n$ elements allowing for duplicates but disregarding different orderings. Using the \emph{stars and bars method}, it can be shown that this number is given by
\[{n+k-1 \choose k}\]

\subsection{Permutations with repetition}
Permutations with repetition are ordered arrangements of $k$ elements from a set $S$ with $n$ elements where repetition is allowed. The number of permutations with repetition of size $k$ is simply $k^n$.
\subsection{Generate permutations of a string}
\subsection{Generate all bit strings of length n}

\section{Relations}
\subsection{\# of relations}
\subsection{\# of transitive relations}
There is no known closed formula for counting the number of transitive relations. The (perhaps inefficient) approach taken in this algorithm is as follows
\begin{enumerate}
\item[(1)] Generate all possible relations for an $n$ element set (given by the power set of the cartesian product of the set $\{1,2,3,\dots,n\}$)
\item[(2)] For each relation generated in (1), check that for each $(a,b)$, if there is a point of the form $(b,c)$, then $(a,c)$ must be in the relation
\end{enumerate}
\subsection{\# of (ir)reflexive relations}
\subsection{\# of symmetric relations}
\subsection{\# of antisymmetric relations}
\subsection{\# of equivalence relations}

\section{Sets}
\subsection{Generate power set}
\subsection{Generate cartesian product}

\section{Isomorphisms}
maybe total orders?

\section{Default}
No documentation provided.


\end{document}
